\documentclass[12pt, a4paper]{article}
\usepackage[utf8]{inputenc}
\usepackage{amsmath}
\usepackage{xcolor}
\usepackage{amssymb}
\usepackage{multicol}
\usepackage{changepage}
\date{$5^{th}$ January, 2022}
\title{Math 3 Equations Revision}
\author{Muhammed Abdullsalam}
\numberwithin{equation}{section}

% Columns separation %
\setlength{\columnsep}{0.5cm}
\setlength{\columnseprule}{1pt}
\def\columnseprulecolor{\color{darkgray}}


\begin{document}
	\maketitle
	\tableofcontents
	\pagebreak
	
	\section{Laplace Transforms}
	\label{sec_laplace}
	\subsection{Laplace General Equation}
	General equation:
	\begin{equation}
		L\left[f(t)\right] = \int_{0}^{\infty}f(t) e^{-st}\,dt = F(s)
		\label{eq_laplace}
	\end{equation}
	\\\\
	\subsection{Basic Laplace Transforms}
	Basic transforms for Laplace:\\
	
	\textbf{Constants}
	\begin{equation}
		L[c] = \int_{0}^{\infty}c e^{-st}\,dt = \frac{c}{s}
		\label{eq_laplace_trans_const}
	\end{equation}
	
	\textbf{Euler's Number Exponents}
	\begin{equation}
		L[e^{at}] = \int_{0}^{\infty}e^{at} e^{-st}\,dt = \frac{1}{s - a}
		\label{eq_laplace_trans_euler_exponent}
	\end{equation}
	
	\textbf{'t' Exponents}
	\begin{equation}
		L[t^n] = \frac{n!}{s^{n+1}}
		\label{eq_laplace_trans_t_exponent}
	\end{equation}
	
	\textbf{Trigonometric Functions}
	\begin{equation}
		L[\sin at] = \int_{0}^{\infty}\sin(at) e^{-st}\,dt = \frac{a}{s^2 + a^2}
		\label{eq_laplace_trans_trig_sin}
	\end{equation}
	\begin{equation}
		L[\cos at] = \int_{0}^{\infty}\cos(at) e^{-st}\,dt = \frac{s}{s^2 + a^2}
		\label{eq_laplace_trans_trig_cos}
	\end{equation}
	\begin{equation}
		L[\sinh at] = \int_{0}^{\infty}\sinh(at) e^{-st}\,dt = \frac{a}{s^2 - a^2}
		\label{eq_laplace_trans_trig_sinh}
	\end{equation}
	\begin{equation}
		L[\cosh at] = \int_{0}^{\infty}\cosh(at) e^{-st}\,dt = \frac{s}{s^2 - a^2}
		\label{eq_laplace_trans_trig_cosh}
	\end{equation}
	
	\subsection{Miscellaneous}
	Combinations of basic functions:\\
	
	\textbf{Euler's Number Exponents $\times$ Trig. Functions}
	\begin{equation}
		L[e^{wt} \cos at] = \int_{0}^{\infty}e^{wt}\cos(at) e^{-st\,dx} = \int_{0}^{\infty}\cos(at) e^{(w - s)t} = \frac{s - w}{(s - 2)^2 + a^2}
		\label{eq_laplace_trans_trig_cos_euler}
	\end{equation}
	\begin{equation}
		L[e^{wt} \sin at] = \int_{0}^{\infty}e^{wt}\sin(at) e^{-st\,dx} = \int_{0}^{\infty}\sin(at) e^{(w - s)t} = \frac{a}{(s - 2)^2 + a^2}
		\label{eq_laplace_trans_trig_sin_euler}
	\end{equation}
	\begin{equation}
		L[e^{wt} \sinh at] = \int_{0}^{\infty}e^{wt}\sinh(at) e^{-st\,dx} = \int_{0}^{\infty}\sinh(at) e^{(w - s)t} = \frac{a}{(s - 2)^2 - a^2}
		\label{eq_laplace_trans_trig_sinh_euler}
	\end{equation}
	\begin{equation}
		L[e^{wt} \cosh at] = \int_{0}^{\infty}e^{wt}\cosh(at) e^{-st\,dx} = \int_{0}^{\infty}\cosh(at) e^{(w - s)t} = \frac{s - w}{(s - 2)^2 - a^2}
		\label{eq_laplace_trans_trig_cosh_euler}
	\end{equation}
	
	\textbf{A Function $\times$ 't' Exponents}
	\begin{equation}
		L[t^n f(t)] = (-1)^n \frac{d^n}{ds^n} L[f(t)]
		\label{eq_laplace_trans_function_t}
	\end{equation}
	
	\subsection{Comprehensive Example}
	Find Laplace transform of f(t):\\\\
	\begin{math}
		f(t) = t^4 + 3\hspace{0.1cm} e^{-6t} \cosh(\sqrt{2}t) + 2\hspace{0.1cm} t^2 \sin(5t)
	\end{math}\\
	
	\textbf{Solution}:\\
	From equations (\ref{eq_laplace_trans_t_exponent}) \& (\ref{eq_laplace_trans_trig_cosh_euler}) \& (\ref{eq_laplace_trans_function_t}) :\\
	{\color{darkgray}
		\begin{flalign}
			\nonumber
			L[f(t)] &= L[t^4] + 3\hspace{0.1cm} L[e^{-6t} \cosh(\sqrt{2}t)] + 2\hspace{0.1cm} L[t^2 \sin(5t)]&\\ \nonumber \\ \nonumber
			&= \frac{4!}{s^{4+1}} + 3\hspace{0.1cm} \frac{s + 6}{(s + 6)^2 - {\sqrt{2}}^2} + 2\hspace{0.1cm} (-1)^2 \frac{d^2}{ds^2} L[\sin(5t)]&\\ \nonumber \\ \nonumber
			&= \frac{24}{s^5} + 3\hspace{0.1cm}\frac{s + 6}{(s + 6)^2 - 2} + 2\hspace{0.1cm}\frac{d^2}{ds^2}\left(\frac{5}{s^2 + 25}\right)&\\ \nonumber \\ \nonumber
			&= \frac{24}{s^5} + 3\hspace{0.1cm}\frac{s + 6}{(s + 6)^2 - 2} + 2\hspace{0.1cm}\frac{d}{ds}\left(\frac{0 \times (s^2 + 25) - 2s \times 5}{(s^2 + 25)^2}\right)&\\ \nonumber \\ \nonumber
			&= \frac{24}{s^5} + 3\hspace{0.1cm}\frac{s + 6}{(s + 6)^2 - 2} + 2(-10)\hspace{0.1cm}\frac{d}{ds}\left(\frac{s}{(s^2 + 25)^2}\right)&\\ \nonumber \\ \nonumber
			&= \frac{24}{s^5} + 3\hspace{0.1cm}\frac{s + 6}{(s + 6)^2 - 2} - 20\hspace{0.1cm}\frac{(s^2 + 25)^2 - s \times 2(2s(s^2 + 25))}{(s^2 + 25)^4}&\\ \nonumber \\ \nonumber
			&= \frac{24}{s^5} + 3\hspace{0.1cm}\frac{s + 6}{(s + 6)^2 - 2} - 20\hspace{0.1cm}\frac{(s^2 + 25)^2 - 4s^2 (s^2 + 25)}{(s^2 + 25)^4}&\\ \nonumber \\ \nonumber
			&= \frac{24}{s^5} + 3\hspace{0.1cm}\frac{s + 6}{(s + 6)^2 - 2} - 20\hspace{0.1cm}\frac{((s^2 + 25))((s^2 + 25) - 4s^2)}{(s^2 + 25)^4}&\\ \nonumber \\ \nonumber
			&= \frac{24}{s^5} + 3\hspace{0.1cm}\frac{s + 6}{(s + 6)^2 - 2} - 20\hspace{0.1cm}\frac{s^2 + 25 - 4s^2}{(s^2 + 25)^3}&\\ \nonumber \\ \nonumber
			&= \frac{24}{s^5} + 3\hspace{0.1cm}\frac{s + 6}{(s + 6)^2 - 2} - 20\hspace{0.1cm}\frac{25 - 3s^2}{(s^2 + 25)^3}&\\ \nonumber \\ \nonumber
			&= \frac{24}{s^5} + 3\hspace{0.1cm}\frac{s + 6}{(s + 6)^2 - 2} + 20\hspace{0.1cm}\frac{3s^2 - 25}{(s^2 + 25)^3}
		\end{flalign}
	}
	
	\pagebreak
	
	\section{Inverse Laplace Transforms}
	\label{sec_inverse_laplace}
	You can obtain inverse Laplace transform using same equations of section \ref{sec_laplace}.\\
	
	\begin{equation}
		f(t) = L^{-1}[F(s)]
		\label{eq_inverse_laplace}
	\end{equation}\\
	
	Only thing you need to do is to manipulate and simplify your equation to looks like one of the right hand side part of equations in section \ref{sec_laplace}, then its inverse Laplace is the left hand side part of the same equation
	
	\subsection{Inverse Laplace Equations}
	Basic transforms for Laplace:\\
	
	\textbf{Constants}
	\begin{equation}
		L^{-1}\left[\frac{c}{s}\right] = c
		\label{eq_inverse_laplace_trans_const}
	\end{equation}
	
	\textbf{Euler's Number Exponents}
	\begin{equation}
		L^{-1}\left[\frac{1}{s - a}\right] = e^{at}
		\label{eq_inverse_laplace_trans_euler_exponent}
	\end{equation}
	
	\textbf{'t' Exponents}
	\begin{equation}
		L^{-1}\left[\frac{n!}{s^{n+1}}\right] = t^n
		\label{eq_inverse_laplace_trans_t_exponent}
	\end{equation}
	
	\textbf{Trigonometric Functions}
	\begin{equation}
		L^{-1}\left[\frac{a}{s^2 + a^2}\right] = \sin at
		\label{eq_inverse_laplace_trans_trig_sin}
	\end{equation}
	\begin{equation}
		L^{-1}\left[\frac{s}{s^2 + a^2}\right] = \cos at
		\label{eq_inverse_laplace_trans_trig_cos}
	\end{equation}
	\begin{equation}
		L^{-1}\left[\frac{a}{s^2 - a^2}\right]  = \sinh at
		\label{eq_inverse_laplace_trans_trig_sinh}
	\end{equation}
	\begin{equation}
		L^{-1}\left[\frac{s}{s^2 - a^2}\right]  = \cosh at
		\label{eq_inverse_laplace_trans_trig_cosh}
	\end{equation}
	
	\textbf{Euler's Number Exponents $\times$ Trig. Functions}
	\begin{equation}
		L^{-1}\left[\frac{s - w}{(s - 2)^2 + a^2}\right] = e^{wt} \cos at
		\label{eq_inverse_laplace_trans_trig_cos_euler}
	\end{equation}
	\begin{equation}
		L^{-1}\left[\frac{a}{(s - 2)^2 + a^2}\right] = e^{wt} \sin at
		\label{eq_inverse_laplace_trans_trig_sin_euler}
	\end{equation}
	\begin{equation}
		L^{-1}\left[\frac{a}{(s - 2)^2 - a^2}\right] = e^{wt} \sinh at
		\label{eq_inverse_laplace_trans_trig_sinh_euler}
	\end{equation}
	\begin{equation}
		L^{-1}\left[\frac{s - w}{(s - 2)^2 - a^2}\right] = e^{wt} \cosh at
		\label{eq_inverse_laplace_trans_trig_cosh_euler}
	\end{equation}
	
	\textbf{A Function $\times$ 't' Exponents}
	\begin{equation}
		L^{-1}[F^{(n)}(s)] = (-1)^n \hspace{0.1cm} t^n \hspace{0.1cm} f(t)
		\label{eq_inverse_laplace_trans_function_t}
	\end{equation}\\

	\textbf{Conclusion}: You can see that each Laplace transform equation in section \ref{sec_inverse_laplace} has its corresponding inverse Laplace transform equation in section \ref{sec_laplace}.\\
	So that equation 1.X has corresponding inverse equation 2.X .
	
	\pagebreak
	
	\subsection{Revision on Partial Fractions}
	Before proceeding to any example we must revise some essential rules of partial fractions...\\
	
	The theory of partial fractions applies chiefly to the ratio of two polynomials in which the degree of the numerator is strictly less than that of the denominator. Such a ratio is called a “proper rational function”. \cite{Hobson_2002}\\\\
	For a rational function which is not proper, it is necessary first to use long division of
	polynomials in order to express it as the sum of a polynomial and a proper rational function.\\
	
	\textbf{Linear Factors in Denominator}
		\begin{flalign}
			\frac{7x + 8}{(2x + 3)(x - 1)} \equiv \frac{A}{2x + 3} + \frac{B}{x - 1}
			\label{eq_partial_frac_1}
		\end{flalign}\\
		Multiplying by $(2x + 3)(x - 1)$, we obtain
		\begin{flalign}\nonumber
			7x + 8 \equiv A(x - 1) + B(2x + 3)
		\end{flalign}
		let $x = 1$,
		\begin{flalign}\nonumber
			7 + 8 &= A(1 - 1) + B(2 + 3)\\ \nonumber
			7 + 8 &= B(2 + 3)\\ \nonumber
			B &= \frac{7 + 8}{2 + 3} = \frac{15}{5} = 3
		\end{flalign}
	\\
		let $x = -\frac{3}{2}$,
		\begin{flalign}\nonumber
			7(\frac{-3}{2}) + 8 &= A(\frac{-3}{2} - 1) + B(2(\frac{-3}{2}) + 3)\\ \nonumber
			\frac{-21}{2} + 8 &= A(\frac{-3}{2} - 1)\\ \nonumber
			A &= \frac{\frac{-21}{2} + 8}{\frac{-3}{2} - 1} = 1
		\end{flalign}
		\begin{flalign}\nonumber
			\frac{7x + 8}{(2x + 3)(x - 1)} = \frac{1}{2x + 3} + \frac{3}{x - 1}
		\end{flalign}\\
	
	\textbf{One Linear Factor \& One Quadratic Factor in Denominator}\\\\
	We should observe firstly that the quadratic factor will not reduce conveniently into two
	linear factors. If it did, the method would be as in the previous paragraph. \cite{Hobson_2002}
	\begin{flalign}
		\frac{3x^2 + 9}{(x - 5)(x^2 + 2x + 7)} \equiv \frac{A}{x - 5} + \frac{Bx + C}{x^2 + 2x + 7}
		\label{eq_partial_frac_2}
	\end{flalign}\\
	Multiplying by $(x - 5)(x^2 + 2x + 7)$, we obtain
	\begin{flalign}\nonumber
		3x^2 + 9 \equiv A(x^2 + 2x + 7) + (Bx + C)(x - 5)
	\end{flalign}
	let $x = 5$,
	\begin{flalign}\nonumber
		3(5)^2 + 9 &= A(5^2 + 2(5) + 7) + (Bx + C)(5 - 5)\\ \nonumber
		75 + 9 &= A(25 + 10 + 7)\\ \nonumber
		A &= \frac{84}{42} =  2
	\end{flalign}
	
	No other convenient values of $x$ may be substituted; but two polynomial expressions can be identical only if their corresponding coefficients are the same in value. We therefore equate suitable coefficients to find B and C; usually, the coefficients of the highest and lowest powers
	of $x$. \cite{Hobson_2002}\\\\
	Equating coefficients of $x^2$, $3 = A + B$ and hence $B = 1$.\\\\
	Equating constant terms (the coefficients of $x^0$), $9 = 7A - 5C = 14 - 5C$ and hence $C = 1$.
	\begin{flalign}\nonumber
		\frac{3x^2 + 9}{(x - 5)(x^2 + 2x + 7)} = \frac{2}{x - 5} + \frac{x + 1}{x^2 + 2x + 7}
	\end{flalign}
	\pagebreak
	
	\textbf{Repeated Linear Factor in Denominator}
	\begin{flalign}
		\frac{9}{(x + 1)^2(x - 2)} \equiv \frac{A}{x + 1} + \frac{C}{(x + 1)^2} + \frac{D}{x - 2}
		\label{eq_partial_frac_3}
	\end{flalign}\\
	Eliminating fractions, we obtain
	\begin{flalign}\nonumber
		9 \equiv A(x + 1)(x - 2) + C(x - 2) + D(x + 1)^2
	\end{flalign}\\
	Putting $x = -1$ gives $9 = -3C$ so that $C = -3$.\\\\
	Putting $x = 2$ gives $9 = 9D$ so that $D = 1$.\\\\
	Equating coefficients of $x^2$ gives $0 = A + D$ so that $A = -1$.
	\begin{flalign}\nonumber
		\frac{9}{(x + 1)^2(x - 2)} = \frac{-1}{x + 1} - \frac{3}{(x + 1)^2} + \frac{1}{x - 2}
	\end{flalign}
	\pagebreak
	
	\subsection{Examples}
	Find Inverse Laplace of F(s):
	\begin{flalign}\nonumber
		1.\hspace{0.3cm} &F(s) = \frac{s + 7}{s^2 - 3s - 10}&
	\end{flalign}\\
	
	\textbf{Solution}:\\
	%From equations (\ref{eq_laplace_trans_t_exponent}) \& (\ref{eq_laplace_trans_trig_cosh_euler}) \& (\ref{eq_laplace_trans_function_t}) :\\
	By Factorizing the denominator :
	{\color{darkgray}
		\begin{flalign}\nonumber
			\frac{s + 7}{s^2 - 3s - 10} = \frac{s + 7}{(s + 2) (s - 5)}
		\end{flalign}
	}
	From equation (\ref{eq_partial_frac_1}) :\\
	{\color{darkgray}
		\begin{flalign}\nonumber
			\frac{s + 7}{(s + 2) (s - 5)} &= \frac{A}{s + 2} + \frac{B}{s - 5}
		\end{flalign}
	}
	Multiplying by $(s + 2) (s - 5)$, we obtain
	{\color{darkgray}
		\begin{flalign}\nonumber
			s + 7 &= A(s - 5) + B(s + 2)
		\end{flalign}
	}
	let $s = 5$
	{\color{darkgray}
		\begin{flalign}\nonumber
			&5 + 7 = A(5 - 5) + B(5 + 2)\\ \nonumber \\ \nonumber
			&5 + 7 = B(5 + 2)\\ \nonumber \\ \nonumber
			&B = \frac{5 + 7}{5 + 2} = \frac{12}{7}
		\end{flalign}
	}
	let $s = -2$
	{\color{darkgray}
		\begin{flalign}\nonumber
			-&2 + 7 = A(-2 - 5) + B(-2 + 2)\\ \nonumber \\ \nonumber
			-&2 + 7 = A(-2 - 5)\\ \nonumber \\ \nonumber
			&A = \frac{-2 + 7}{-2 - 5} = -\frac{5}{7}
		\end{flalign}
	}

	From equation (\ref{eq_inverse_laplace_trans_euler_exponent}) :\\
	{\color{darkgray}
		\begin{flalign}\nonumber
			&F(s) = (-\frac{5}{7})\frac{1}{s + 2} + (\frac{12}{7})\frac{1}{s - 5} \\ \nonumber \\ \nonumber
			&L^{-1}[F(s)] = -\frac{5}{7}\hspace{0.1cm} e^{-2t} + \frac{12}{7}\hspace{0.1cm} e^{5t}
		\end{flalign}
	}
\\

	\begin{flalign}\nonumber
		2.\hspace{0.3cm} &F(s) = \frac{3s - 2}{2s^2 - 6s - 2}&
	\end{flalign}\\

	\textbf{Solution}:
	%From equations (\ref{eq_laplace_trans_t_exponent}) \& (\ref{eq_laplace_trans_trig_cosh_euler}) \& (\ref{eq_laplace_trans_function_t}) :\\
	{\color{darkgray}
		\begin{multicols}{2}
		\begin{flalign}\nonumber
			F(s) &= (1)\frac{3s - 2}{2s^2 - 6s - 2}\\ \nonumber \\ \nonumber
			&= (\frac{2}{2})\frac{3s - 2}{2s^2 - 6s - 2}\\ \nonumber \\ \nonumber
			&= \frac{\frac{3}{2}\hspace{0.1cm}s - \frac{2}{2}}{\frac{2}{2}\hspace{0.1cm}s^2 - \frac{6}{2}\hspace{0.1cm}s - \frac{2}{2}}\\ \nonumber \\ \nonumber
			&= \frac{\frac{3}{2}\hspace{0.1cm}s - 1}{s^2 - 3s - 1}\\ \nonumber \\ \nonumber
			&= \frac{\frac{3}{2}\hspace{0.1cm}s - 1}{s^2 - 3s - 1 + 0}\\ \nonumber \\ \nonumber
			&= \frac{\frac{3}{2}\hspace{0.1cm}s - 1}{s^2 - 3s - 1 + ((\frac{3}{2})^2 - (\frac{3}{2})^2)}
		\end{flalign}
		\columnbreak
		\begin{flalign}\nonumber \\ \nonumber \\ \nonumber \\ \nonumber
			&= \frac{\frac{3}{2}\hspace{0.1cm}s - 1}{s^2 - 3s - 1 + \frac{9}{4} - \frac{9}{4}}\\ \nonumber \\ \nonumber
			&= \frac{\frac{3}{2}\hspace{0.1cm}s - 1}{s^2 - 3s + \frac{9}{4} - 1 - \frac{9}{4}}\\ \nonumber \\ \nonumber
			&= \frac{\frac{3}{2}\hspace{0.1cm}s - 1}{(s - \frac{3}{2})^2 - \frac{13}{4}} \\ \nonumber \\ \nonumber
			&= \frac{\frac{3}{2}\hspace{0.1cm}s}{(s - \frac{3}{2})^2 - \frac{13}{4}} - \frac{1}{(s - \frac{3}{2})^2 - \frac{13}{4}} \\ \nonumber \\ \nonumber
			&= \frac{3}{2}\frac{s - \frac{3}{2} + \frac{3}{2}}{(s - \frac{3}{2})^2 - \frac{13}{4}} - \frac{1}{(s - \frac{3}{2})^2 - \frac{13}{4}}
		\end{flalign}
		\end{multicols}
		\begin{flalign}\nonumber
			&= \frac{3}{2}\left(\frac{s - \frac{3}{2}}{(s - \frac{3}{2})^2 - \frac{13}{4}} + \frac{\frac{3}{2}}{(s - \frac{3}{2})^2 - \frac{13}{4}}\right) - \frac{1}{(s - \frac{3}{2})^2 - \frac{13}{4}} \\ \nonumber \\ \nonumber
			&= \frac{3}{2}\frac{s - \frac{3}{2}}{(s - \frac{3}{2})^2 - \frac{13}{4}} + \frac{3}{2}\frac{\frac{3}{2}}{(s - \frac{3}{2})^2 - \frac{13}{4}} - \frac{1}{(s - \frac{3}{2})^2 - \frac{13}{4}} \\ \nonumber \\ \nonumber
			&= \frac{3}{2}\frac{s - \frac{3}{2}}{(s - \frac{3}{2})^2 - \frac{13}{4}} + \frac{9}{4}\frac{1}{(s - \frac{3}{2})^2 - \frac{13}{4}} - \frac{1}{(s - \frac{3}{2})^2 - \frac{13}{4}} \\ \nonumber \\ \nonumber
			&= \frac{3}{2}\frac{s - \frac{3}{2}}{(s - \frac{3}{2})^2 - \frac{13}{4}} + \frac{9}{4}\times\frac{2}{\sqrt{13}}\frac{\frac{\sqrt{13}}{2}}{(s - \frac{3}{2})^2 - \frac{13}{4}} - \frac{2}{\sqrt{13}}\frac{\frac{\sqrt{13}}{2}}{(s - \frac{3}{2})^2 - \frac{13}{4}} \\ \nonumber \\ \nonumber
			&= \frac{3}{2}\frac{s - \frac{3}{2}}{(s - \frac{3}{2})^2 - \frac{13}{4}} + \frac{9\sqrt{13}}{26}\frac{\frac{\sqrt{13}}{2}}{(s - \frac{3}{2})^2 - \frac{13}{4}} - \frac{2}{\sqrt{13}}\frac{\frac{\sqrt{13}}{2}}{(s - \frac{3}{2})^2 - \frac{13}{4}} \\ \nonumber \\ \nonumber
			L^{-1}[F(s)] &= \frac{3}{2} \hspace{0.1cm} e^{\frac{3}{2}t}\hspace{0.1cm} \cosh(\frac{\sqrt{13}}{2}\hspace{0.1cm}t) + \frac{9\sqrt{13}}{26} \hspace{0.1cm} e^{\frac{3}{2}t}\hspace{0.1cm} \sinh(\frac{\sqrt{13}}{2}\hspace{0.1cm}t) - \frac{2}{\sqrt{13}} \hspace{0.1cm} e^{\frac{3}{2}t}\hspace{0.1cm} \sinh(\frac{\sqrt{13}}{2}\hspace{0.1cm}t)\\ \nonumber \\ \nonumber
			&= \frac{3}{2} \hspace{0.1cm} e^{\frac{3}{2}t}\hspace{0.1cm} \cosh(\frac{\sqrt{13}}{2}\hspace{0.1cm}t) + \frac{5\sqrt{13}}{26} \hspace{0.1cm} e^{\frac{3}{2}t}\hspace{0.1cm} \sinh(\frac{\sqrt{13}}{2}\hspace{0.1cm}t)
		\end{flalign}
	}
	\pagebreak

	\section{Fourier Series}
		\label{sec_fourier_series}
		\textbf{\Large F}ourier series is a periodic function composed of harmonically related sinusoids combined by a weighted summation. \cite{wikipedia_fourier_series}

	\subsection{Revision on Some Trig. Functions}

		\begin{equation}
			\sin(a) \sin(b) = \frac{1}{2}\left[\cos(a - b) - \cos(a + b)\right]
			\label{eq_trig_1}
		\end{equation}
	
		\begin{equation}
		\cos(a) \cos(b) = \frac{1}{2}\left[\cos(a - b) + \cos(a + b)\right]
		\label{eq_trig_2}
		\end{equation}
	
		\begin{equation}
		\sin(a) \cos(b) = \frac{1}{2}\left[\sin(a - b) + \sin(a + b)\right]
		\label{eq_trig_3}
		\end{equation}\\
	
		\subsection{Definite Integrals of Some Combined Trig. Functions}
		Some of the following results can be proven using equations (\ref{eq_trig_1}) (\ref{eq_trig_2}) (\ref{eq_trig_3})\\
		
		For any integer m
		\begin{equation}
			\int_{-\pi}^{\pi}\sin(mt)\,dt = 0
			\label{eq_def_integral_sin}
		\end{equation}

		For non-zero integer m
		\begin{equation}
			\int_{-\pi}^{\pi}\cos(mt)\,dt = 0
			\label{eq_def_integral_cos}
		\end{equation}
	
		For any integers m, n
		\begin{equation}
			\int_{-\pi}^{\pi}\sin(mt)\cos(nt)\,dt = 0
			\label{eq_def_integral_sin_cos}
		\end{equation}
	
		When integers m $\neq$ n
		\begin{equation}
			\int_{-\pi}^{\pi}\sin(mt)\sin(nt)\,dt = 0
			\label{eq_def_integral_sin_sin}
		\end{equation}
	
		When m is non-zero int
		\begin{equation}
			\int_{-\pi}^{\pi}\sin^2(mt)\,dt = \pi
			\label{eq_def_integral_sin^2}
		\end{equation}
	
		When integers m $\neq$ n
		\begin{equation}
			\int_{-\pi}^{\pi}\cos(mt)\cos(nt)\,dt = 0
			\label{eq_def_integral_cos_cos}
		\end{equation}
		
		When m is non-zero int
		\begin{equation}
			\int_{-\pi}^{\pi}\cos^2(mt)\,dt = \pi
			\label{eq_def_integral_cos^2}
		\end{equation}\\
		

		\subsection{Fourier Series General Formula}
		A simple description of Fourier series is expressing a function as sum of weighted sine and cosine terms.\\
		
		If we have a function $f(t)$, we can express it like this 
		\begin{flalign}\nonumber
			f(t) = &a_0 \cos(0t) + a_1 \cos(t) + a_2 \cos(2t)  + a_3 \cos(3t) + ...\\ \nonumber
			&b_0 \sin(0t) + b_1 \sin(t) + b_2 \sin(2t) + b_3 \sin(3t) + ...
		\end{flalign}
		\begin{flalign}\nonumber
			= a_0 + &a_1 \cos(t) + a_2 \cos(2t)  + a_3 \cos(3t) + ...\\ \nonumber
			&b_1 \sin(t) + b_2 \sin(2t) + b_3 \sin(3t) + ...
		\end{flalign}\\
		
		We can express this in a neater way
		\begin{equation}
			f(t) = a_0 + \sum_{n = 1}^\infty a_n \cos(nt) + b_n \sin(nt)
			\label{eq_fourier_general}
		\end{equation}

		\textbf{Finding First Term $(a_0)$}
		\begin{flalign}\nonumber
			\int_{-\pi}^{\pi}f(t)\,dt = \int_{-\pi}^{\pi} a_0 \,dt + \sum_{n = 1}^\infty \int_{-\pi}^{\pi} a_n \cos(nt) \,dt + \int_{-\pi}^{\pi} b_n \sin(nt) \,dt
		\end{flalign}
		From equations (\ref{eq_def_integral_sin}) \& (\ref{eq_def_integral_cos})
		\begin{flalign}\nonumber
			\int_{-\pi}^{\pi}f(t) \,dt &= \int_{-\pi}^{\pi} a_0 \,dt + \sum_{n = 1}^\infty 0 + 0\\ \nonumber
			&= \int_{-\pi}^{\pi} a_0 \,dt\\ \nonumber
			&= a_0 [t]|_{-\pi}^{\pi}\\ \nonumber
			&= a_0 [\pi - (-\pi)]\\ \nonumber
			&= a_0 [\pi + \pi]\\ \nonumber
			&= a_0 (2\pi)
		\end{flalign}
	
		\begin{equation}
			a_0 = \frac{1}{2\pi} \int_{-\pi}^{\pi}f(t) \,dt
			\label{eq_fourier_first_term}
		\end{equation}
		\\
	
	\textbf{Finding General Equation Coefficients for Cosine Terms} $(a_n)$
		\begin{adjustwidth}{-50pt}{0pt}
		\begin{flalign}\nonumber
			\int_{-\pi}^{\pi}f(t) \cos(mt)\,dt = \int_{-\pi}^{\pi} a_0 \cos(mt) \,dt + \sum_{n = 1}^\infty \int_{-\pi}^{\pi} a_n \cos(nt) \cos(mt) \,dt + \int_{-\pi}^{\pi} b_n \sin(nt) \cos(mt) \,dt
		\end{flalign}
		\end{adjustwidth}
		From equations (\ref{eq_def_integral_cos}) (\ref{eq_def_integral_cos_cos}) (\ref{eq_def_integral_sin_cos}) all terms will equal to zero except cosine term where $m = n$
		\begin{flalign*}
			\int_{-\pi}^{\pi}f(t) \cos(mt)\,dt &= \int_{-\pi}^{\pi} a_n \cos(nt) \cos(mt)\\
			= \int_{-\pi}^{\pi}f(t) \cos(nt)\,dt &= \int_{-\pi}^{\pi} a_n \cos^2(nt)
		\end{flalign*}
	
		From equation (\ref{eq_def_integral_cos^2})
		\begin{flalign}\nonumber
			&\int_{-\pi}^{\pi}f(t) \cos(nt)\,dt = a_n\hspace{0.1cm} \pi
		\end{flalign}
	
		\begin{equation}
			a_n = \frac{1}{\pi} \int_{-\pi}^{\pi} f(t) \cos(nt)\,dt
			\label{eq_fourier_cos_coeff}
		\end{equation}\pagebreak
		
		\textbf{Finding General Equation Coefficients for Sine Terms} $(b_n)$
		\begin{adjustwidth}{-50pt}{0pt}
			\begin{flalign}\nonumber
				\int_{-\pi}^{\pi}f(t) \sin(mt)\,dt = \int_{-\pi}^{\pi} a_0 \sin(mt) \,dt + \sum_{n = 1}^\infty \int_{-\pi}^{\pi} a_n \cos(nt) \sin(mt) \,dt + \int_{-\pi}^{\pi} b_n \sin(nt) \sin(mt) \,dt
			\end{flalign}
		\end{adjustwidth}
		From equations (\ref{eq_def_integral_sin}) (\ref{eq_def_integral_sin_sin}) (\ref{eq_def_integral_sin_cos}) all terms will equal to zero except sine term where $m = n$
		\begin{flalign*}
			\int_{-\pi}^{\pi}f(t) \sin(mt)\,dt &= \int_{-\pi}^{\pi} b_n \sin(nt) \sin(mt)\\
			= \int_{-\pi}^{\pi}f(t) \sin(nt)\,dt &= \int_{-\pi}^{\pi} b_n \sin^2(nt)
		\end{flalign*}
		
		From equation (\ref{eq_def_integral_sin^2})
		\begin{flalign}\nonumber
			&\int_{-\pi}^{\pi}f(t) \sin(nt)\,dt = b_n\hspace{0.1cm} \pi
		\end{flalign}
	
		\begin{equation}
				b_n = \frac{1}{\pi} \int_{-\pi}^{\pi} f(t) \sin(nt)\,dt
				\label{eq_fourier_sin_coeff}
		\end{equation}\\

		\textbf{Formula for Other Intervals}
		
		If you try to apply Fourier transform on a function from interval $\left[-L, L\right]$ then the equations will slightly change
		
		\begin{equation}
			a_0 = \frac{1}{2L} \int_{-L}^{L}f(t) \,dt
			\label{eq_fourier_first_term_other_interval}
		\end{equation}
		
		\begin{equation}
			a_n = \frac{1}{L} \int_{-L}^{L} f(t) \cos\left(\frac{n \pi t}{L}\right)\,dt
			\label{eq_fourier_cos_coeff_other_interval}
		\end{equation}
		
		\begin{equation}
			b_n = \frac{1}{L} \int_{-L}^{L} f(t) \sin\left(\frac{n \pi t}{L}\right)\,dt
			\label{eq_fourier_sin_coeff_other_interval}
		\end{equation}

		\pagebreak
		
	\subsection{Examples}
	Find Fourier series of $f(x)$:
	\begin{flalign}\nonumber
		1.\hspace{0.3cm} &f(x) = x\hspace{0.1cm} , \hspace{2cm} -\pi \leq x \leq \pi&
	\end{flalign}\\

	\textbf{Solution}:\\
	
	From equation (\ref{eq_fourier_general}), we can write
	{\color{darkgray}
		\begin{flalign*}
			f(x) = a_0 + \sum_{n = 1}^\infty a_n \cos(nt) + b_n \sin(nt)
		\end{flalign*}
	}
	Lets find the first term as in equation (\ref{eq_fourier_first_term})
	{\color{darkgray}
		\begin{flalign*}
			a_0 &= \frac{1}{2\pi}\int_{-\pi}^{\pi}f(x)\,dx\\ \\
			&= \frac{1}{2\pi}\int_{-\pi}^{\pi}x\,dx\\ \\
			&= \frac{1}{2\pi}[\frac{1}{2}x^2]|_{-\pi}^{pi}\\ \\
			&= \frac{1}{2\pi}\times\frac{1}{2}[\pi^2 - (-\pi)^2]\\ \\
			&= \frac{1}{4\pi}[\pi^2 - \pi^2]\\ \\
			&= \frac{1}{4\pi}[0]\\ \\
			&= 0
		\end{flalign*}
	}
	\\ \\ \\
	Finding the general equation coefficient of cosine terms from equation (\ref{eq_fourier_cos_coeff})
	{\color{darkgray}
		\begin{flalign*}
			a_n &= \frac{1}{\pi}\int_{-\pi}^{\pi}f(x)\cos(nx)\,dx\\ \\
			&= \frac{1}{\pi}\int_{-\pi}^{\pi}x\cos(nx)\,dx
		\end{flalign*}
	}

	Lets recall how to integrate by parts:	\\
	To summarize, the formula for “integration by parts”\cite{Hobson_2002}
	\begin{flalign*}
		\int u\hspace{0.1cm}\frac{dv}{dx}\,dx = uv - \int v\hspace{0.1cm}\frac{du}{dx}dx
	\end{flalign*}

	So ,
	{\color{darkgray}
		\begin{flalign*}
			a_n &= \frac{1}{\pi}\left[\frac{x}{n}\sin(nx) - \frac{1}{n}\int_{-\pi}^{\pi}\sin(nx)\,dx\right]_{-\pi}^{\pi}\\ \\
			&= \frac{1}{\pi}\left[\frac{x}{n}\sin(nx) - \frac{1}{n}\left[-\frac{1}{n}\cos(nx)\right]_{-\pi}^{\pi}\right]_{-\pi}^{\pi}\\ \\
			&= \frac{1}{\pi}\left[\frac{x}{n}\sin(nx) + \frac{1}{n^2}\left[\cos(nx)\right]_{-\pi}^{\pi}\right]_{-\pi}^{\pi}
		\end{flalign*}
		\begin{adjustwidth}{-50pt}{0pt}
			\begin{flalign*}
				&= \frac{1}{\pi}\left[ \left(\frac{\pi}{n}\sin(n\pi) + \frac{1}{n^2}\left[\cos(n\pi) - \cos(-n\pi)\right]\right) - \left(\frac{-\pi}{n}\sin(-n\pi) + \frac{1}{n^2}\left[\cos(n\pi) - \cos(-n\pi)\right]\right)\right]
			\end{flalign*}
		\end{adjustwidth}
		\begin{flalign*}
			&= \frac{1}{\pi}\left[ \left(\frac{\pi}{n}\sin(n\pi) + \frac{1}{n^2}\left[0\right]\right) - \left(\frac{-\pi}{n}\sin(-n\pi) + \frac{1}{n^2}\left[0\right]\right)\right]\\ 
			&= \frac{1}{\pi}\left[ \frac{\pi}{n}\sin(n\pi) - \frac{-\pi}{n}\sin(-n\pi)\right]\\ 
			&= \frac{1}{\pi}\left[ \frac{\pi}{n}\sin(n\pi) - \frac{\pi}{n}\sin(n\pi)\right]\\
			&= \frac{1}{\pi}\left[0 - 0\right]\\
			&= \frac{1}{\pi}\left[0\right]\\
			&= 0
		\end{flalign*}
	}
	\\ \\
	Finding the general equation coefficient of cosine terms from equation (\ref{eq_fourier_sin_coeff})
	{\color{darkgray}
		\begin{flalign*}
			b_n &= \frac{1}{\pi}\int_{-\pi}^{\pi}f(x)\sin(nx)\,dx\\ \\
			&= \frac{1}{\pi}\int_{-\pi}^{\pi}x\sin(nx)\,dx
		\end{flalign*}
	}
	Integrate by parts
	\begin{flalign*}
		\int u\hspace{0.1cm}\frac{dv}{dx}\,dx = uv - \int v\hspace{0.1cm}\frac{du}{dx}\,dx
	\end{flalign*}
	{\color{darkgray}
		\begin{flalign*}
			b_n &= \frac{1}{\pi}\left[-\frac{x}{n}\cos(nx) - \frac{-1}{n}\int_{-\pi}^{\pi}\cos(nx)\,dx\right]_{-\pi}^{\pi}\\ \\
			&= \frac{1}{\pi}\left[-\frac{x}{n}\cos(nx) + \frac{1}{n}\left[\frac{1}{n}\sin(nx)\right]_{-\pi}^{\pi}\right]_{-\pi}^{\pi}\\ \\
			&= \frac{1}{\pi}\left[-\frac{x}{n}\cos(nx) + \frac{1}{n^2}\left[\sin(nx)\right]_{-\pi}^{\pi}\right]_{-\pi}^{\pi}
		\end{flalign*}
		\begin{adjustwidth}{-50pt}{0pt}
			\begin{flalign*}
				&= \frac{1}{\pi}\left[ \left(-\frac{\pi}{n}\cos(n\pi) + \frac{1}{n^2}\left[\sin(n\pi) - \sin(-n\pi)\right]\right) - \left(-\frac{-\pi}{n}\cos(-n\pi) + \frac{1}{n^2}\left[\sin(n\pi) - \sin(-n\pi)\right]\right)\right]
			\end{flalign*}
		\end{adjustwidth}
		\begin{flalign*}
			&= \frac{1}{\pi}\left[ \left(-\frac{\pi}{n}\cos(n\pi) + \frac{1}{n^2}\left[0 - 0\right]\right) - \left(-\frac{-\pi}{n}\cos(-n\pi) + \frac{1}{n^2}\left[0 - 0\right]\right)\right]\\
			&= \frac{1}{\pi}\left[ \left(-\frac{\pi}{n}\cos(n\pi) + \frac{1}{n^2}\left[0\right]\right) - \left(-\frac{-\pi}{n}\cos(-n\pi) + \frac{1}{n^2}\left[0\right]\right)\right]\\
			&= \frac{1}{\pi}\left[-\frac{\pi}{n}\cos(n\pi) - \frac{\pi}{n}\cos(-n\pi)\right]\\
			&= \frac{1}{\pi}\left[-\frac{2\pi}{n}\cos(n\pi)\right]\\
		\end{flalign*}		
	}\\ \\

	Lets trace $\cos(n\pi)$ through $n \in \{1, 2, 3, 4, 5\}$
	\begin{multicols}{2}
	\begin{flalign*}
		&\cos(\pi) = -1,\\
		&\cos(2\pi) = 1,\\
		&\cos(3\pi) = -1,
	\end{flalign*}
	\columnbreak
	\begin{flalign*}\\ \\
		\cos(4\pi) &= 1,\\
		\cos(5\pi) &= -1
	\end{flalign*}
	\end{multicols}
	\textbf{Conclusion:} when $n$ is odd the result will be -ve, and when it is even the result will be +ve.\\
	We can express this in a simple way
	\begin{equation}
		\cos(n\pi) = (-1)^n
		\label{eq_cos_n_pi}
	\end{equation}
	so that when $n$ is even the power will neutralize the -ve, otherwise the result will be positive.\\
	
	Lets continue our solution\\
	From equation (\ref{eq_cos_n_pi})
	{\color{darkgray}
		\begin{flalign*}
			b_n &= \frac{1}{\pi}\left[-\frac{2\pi}{n}\cos(n\pi)\right] = \frac{2}{n}(-1)(-1)^n\\
			&= \frac{2}{n}(-1)^{n + 1}
		\end{flalign*}
	}
	So
	{\color{darkgray}
		\begin{flalign*}
			f(x) &= a_0 + \sum_{n = 1}^\infty a_n \cos(nt) + b_n \sin(nt)\\
			&= 0 + \sum_{n = 1}^\infty 0 + (-1)^{n + 1}\hspace{0.1cm}\frac{2}{n} \sin(nt)\\
			&= \sum_{n = 1}^\infty (-1)^{n + 1}\hspace{0.1cm}\frac{2}{n} \sin(nt)\\
			&= (-1)^{1 + 1}\hspace{0.1cm}\frac{2}{1} \sin(t) + (-1)^{2 + 1}\hspace{0.1cm}\frac{2}{2} \sin(2t) + (-1)^{3 + 1}\hspace{0.1cm}\frac{2}{3} \sin(3t) + (-1)^{4 + 1}\hspace{0.1cm}\frac{2}{4} \sin(4t) + ...\\
			&= (-1)^{2}\hspace{0.1cm}2 \sin(t) + (-1)^{3}\hspace{0.1cm}1 \sin(2t) + (-1)^{4}\hspace{0.1cm}\frac{2}{3} \sin(3t) + (-1)^{5}\hspace{0.1cm}\frac{1}{2} \sin(4t) + ...\\
			&= 2 \sin(t) - \sin(2t) + \frac{2}{3} \sin(3t) - \frac{1}{2} \sin(4t) + ...
		\end{flalign*}
	}
	\textbf{\large Odd \& Even Functions}
	
	Note that in previous example the first term $(a_n)$ and the coefficient of cosine terms($a_n$), both equal to $0$.\\
	This gives us an important conclusion: when $f(x)$ is odd we will calculate the terms with odd function only (sine terms), and terms with even function (first term and cosine terms) will equal to $0$.\\\\
	When $f(x)$ is even the terms with odd function (sine terms) will equal to $0$, and terms with even function (first term and cosine terms) will give a value.
	\begin{center}
		(Note that $a_0 = a_0 \cos(0x)$)
	\end{center}
	How do we know that a function is even or that it is odd?\\
	Simply, for a function $f(x)$ when $f(x) = f(-x)$ this function is even function.\\
	If $f(x) = -f(-x)$, this function is odd function. \cite{storyofmathematics.com_2021}\\
	
	Here some of even and odd functions:\\
	\begin{table}[h!]
		\centering
		\begin{tabular}{|c|c|}
			\hline \textbf{\large Even} & \textbf{\large Odd}\\
			\hline $x^2$ & $x$\\
			\hline $x^4$ & $x^3$\\
			\hline $x^6$ & $x^5$\\
			\hline $\cos(x)$ & $\sin(x)$\\
			\hline
		\end{tabular}
		\caption{Some of Even and Odd Functions}
		\label{table_even_odd_func}
	\end{table}
	\pagebreak
	
	\begin{flalign}\nonumber
		2.\hspace{0.3cm} &f(x) = x^2\hspace{0.1cm} , \hspace{2cm} -\pi \leq x \leq \pi&
	\end{flalign}\\
	
	\textbf{Solution}:\\
	{\color{darkgray}
		\begin{flalign*}
			f(x) = a_0 + \sum_{n = 1}^\infty a_n \cos(nt) + b_n \sin(nt)
		\end{flalign*}
	}

	We observe that $f(x)$ is even, So 
	{\color{darkgray}
		\begin{flalign*}
			b_n &= 0,\\ \\
			a_0 &= \frac{1}{2\pi} \int_{-\pi}^{\pi} f(x)\,dx\\
			&= \frac{1}{2\pi} \int_{-\pi}^{\pi} x^2\,dx\\
			&= \frac{1}{2\pi} \left[\frac{1}{3}x^3\right]_{-\pi}^{\pi}\\
			&= \frac{1}{6\pi} \left[(\pi)^3 - (-\pi)^3\right]_{-\pi}^{\pi}\\
			&= \frac{1}{6\pi} \left[\pi^3 + \pi^3\right]\\
			&= \frac{1}{6\pi} \left[2\pi^3\right]\\
			&= \frac{2\pi^3}{6\pi}\\
			&= \frac{1}{3}\hspace{0.1cm}\pi^2\\
		\end{flalign*}
		\begin{flalign*}
			a_n &= \frac{1}{\pi} \int_{-\pi}^{\pi} f(x) cos(nx)\,dx\\
			&= \frac{1}{\pi} \int_{-\pi}^{\pi} x^2 cos(nx)\,dx
		\end{flalign*}
	}
	Integrate by parts
	\begin{flalign*}
		\int u\hspace{0.1cm}\frac{dv}{dx}\,dx = uv - \int v\hspace{0.1cm}\frac{du}{dx}\,dx
	\end{flalign*}
	{\color{darkgray}
		\begin{flalign*}
			&= \frac{1}{\pi} \left[\frac{x^2}{n}\sin(nx) - \frac{2}{n}\int x \sin(nx)\,dx\right]_{-\pi}^{\pi}\\
			&= \frac{1}{\pi} \left[\frac{x^2}{n}\sin(nx) - \frac{2}{n}\left[-\frac{x}{n}\cos(nx) - (-\frac{1}{n}) \int \cos(nx)\,dx\right] \right]_{-\pi}^{\pi}\\
			&= \frac{1}{\pi} \left[\frac{x^2}{n}\sin(nx) + \frac{2}{n^2}\left[x\cos(nx) - \frac{1}{n} \sin(nx)\right] \right]_{-\pi}^{\pi}
		\end{flalign*}
		\begin{adjustwidth}{-90pt}{0pt}
			\begin{flalign*}
				&= \frac{1}{\pi} \left[ \left(\frac{\pi^2}{n}\sin(n\pi) + \frac{2}{n^2}\left[\pi \cos(n\pi) - \frac{1}{n} \sin(n\pi)\right]\right) - \left(\frac{(-\pi)^2}{n}\sin(-n\pi) + \frac{2}{n^2}\left[-\pi \cos(-n\pi) - \frac{1}{n} \sin(-n\pi)\right]\right) \right]\\
			\end{flalign*}
		\end{adjustwidth}
	}
	$\sin(n\pi) = \sin(-n\pi) = 0,$\\
	$\therefore$ all terms have $\sin(n\pi)$ will equal to $0$
	{\color{darkgray}
		\begin{flalign*}
			&= \frac{1}{\pi} \left[ \left(0 + \frac{2}{n^2}\left[\pi \cos(n\pi) - 0\right]\right) - \left(0 + \frac{2}{n^2}\left[-\pi \cos(-n\pi) - 0\right]\right) \right]\\
			&= \frac{1}{\pi} \left[\frac{2}{n^2}\pi \cos(n\pi) - \frac{2}{n^2}-\pi \cos(-n\pi)\right]\\
			&= \frac{1}{\pi} \left[\frac{2}{n^2}\pi \cos(n\pi) + \frac{2\pi}{n^2} \cos(-n\pi)\right]\\
			&= \frac{1}{\pi} (\frac{2}{n^2}) \left[\pi \cos(n\pi) + \pi \cos(-n\pi)\right]\\
			&= \frac{2}{n^2\pi} \left[2 \pi \cos(n\pi)\right]\\
			&= \frac{4\pi}{n^2\pi}\cos(n\pi)\\
			&= \frac{4}{n^2}\cos(n\pi)\\
		\end{flalign*}
	}
	From equation (\ref{eq_cos_n_pi})
	{\color{darkgray}
		\begin{flalign*}
			a_n= (-1)^n \hspace{0.1cm} \frac{4}{n^2}\\
		\end{flalign*}
	}
	\pagebreak
	
	{\color{darkgray}
		\begin{flalign*}
			f(x) &= a_0 + \sum_{n = 1}^\infty a_n \cos(nt) + b_n \sin(nt)\\
			&= \frac{1}{3}\hspace{0.1cm}\pi^2 + \sum_{n = 1}^\infty (-1)^n \hspace{0.1cm} \frac{4}{n^2}\cos(nt) + 0\\
			&= \frac{1}{3}\hspace{0.1cm}\pi^2 + \sum_{n = 1}^\infty (-1)^n \hspace{0.1cm} \frac{4}{n^2}\cos(nt)
		\end{flalign*}
		\begin{adjustwidth}{-80pt}{0pt}
			\begin{flalign*}
				&= \frac{1}{3}\hspace{0.1cm}\pi^2 + (-1)^1 \hspace{0.1cm} \frac{4}{1^2}\cos(t) + (-1)^2 \hspace{0.1cm} \frac{4}{2^2}\cos(2t) + (-1)^3 \hspace{0.1cm} \frac{4}{3^2}\cos(3t) + (-1)^4 \hspace{0.1cm} \frac{4}{4^2}\cos(4t) + (-1)^5 \hspace{0.1cm} \frac{4}{5^2}\cos(5t) + ...
			\end{flalign*}
		\end{adjustwidth}
		\begin{flalign*}
			&= \frac{1}{3}\hspace{0.1cm}\pi^2 - \frac{4}{1}\cos(t) + \frac{4}{4}\cos(2t) - \frac{4}{9}\cos(3t) + \frac{4}{16}\cos(4t) - \frac{4}{25}\cos(5t) + ...\\
			&= \frac{1}{3}\hspace{0.1cm}\pi^2 - 4\cos(t) + 1\cos(2t) - \frac{4}{9}\cos(3t) + \frac{1}{4}\cos(4t) - \frac{4}{25}\cos(5t) + ...\\
		\end{flalign*}
	}
	\pagebreak

	\section{Fourier Transforms}
	\label{sec_fourier_transform}
	\textbf{\Large F}ourier transform is a mathematical transform that decomposes functions depending on space or time into functions depending on frequency. \cite{wikipedia_fourier_transform}
	
	\subsection{Fourier Transform General Equation}
	For $f(t)$, $F(\omega)$ is its Fourier transform\\\\
	\textbf{$f(t)$ is neither even nor odd function}
	\begin{equation}
		F(\omega) = \frac{1}{\sqrt{2\pi}}\int_{-\infty}^{\infty}f(t)\hspace{0.1cm}e^{-i\omega t}\,dt
	\end{equation}
	where $\omega$ is the angular frequency\\\\
	\textbf{$f(t)$ is even function}
	\begin{equation}
		F(\omega) = \frac{2}{\sqrt{2\pi}}\int_{0}^{\infty}f(t)\hspace{0.1cm}\cos(\omega t)\,dt
	\end{equation}
	\textbf{$f(t)$ is odd function}
	\begin{equation}
		F(\omega) = \frac{2}{\sqrt{2\pi}}\int_{0}^{\infty}f(t)\hspace{0.1cm}\sin(\omega t)\,dt
	\end{equation}\\
	\textbf{Some mathematicians prefer this form}
	\begin{equation}
		F(k) = \int_{-\infty}^{\infty}f(x)\hspace{0.1cm}e^{-2\pi i k x}\,dx
	\end{equation}
	and inverse Fourier transform will be
	\begin{equation}
		f(x) = \int_{-\infty}^{\infty}F(k)\hspace{0.1cm}e^{2\pi i k x}\,dk
	\end{equation}
	\pagebreak
	\subsection{Common Fourier Transform Pairs}
	For $f(x)$, both forward Fourier transform and inverse Fourier transform is called its Fourier transform pair.\\
	
	The following table summarize some common of these pairs
	\begin{table}[h!]
		\centering
		\begin{tabular}{|c|c|c|}\hline
			\textbf{Function} & $f(x)$ & $F(k)$\\ \hline
			1 & 1 & $\delta(k)$\\ \hline
			cosine & $\cos(2\pi k_0 x)$ & $\frac{1}{2}[\delta(k - k_0) + \delta(k + k_0)]$\\ \hline
			delta & $\delta(x - x_0)$ & $e^{-2\pi i k x_0}$\\ \hline
			ramp & R(x) & $\pi i \delta(2\pi k) - \frac{1}{4\pi^2 k^2}$\\ \hline
			sine & $\sin(2\pi k_0 x)$ & $\frac{1}{2} i [[\delta(k + k_0) - \delta(k - k_0)]]$\\ \hline
		\end{tabular}
		\caption{Common Fourier transform pairs}
		\label{table_fourier_trans_pairs}
	\end{table}
	
	\subsection{Discrete Fourier Transform}
	Discrete Fourier transforms (DFTs) are extremely useful because they reveal periodicity in input data as well as the relative strengths of any periodic components. \cite{osama_2020} \\
	
	For a discrete function $f(x)$\\
	\textbf{Forward transform}
	\begin{equation}
		F_n = \sum_{k = 0}^{N - 1} f_k e^{-2\pi i n k/N}
	\end{equation}
	\textbf{Inverse transform}
	\begin{equation}
		f_k = \frac{1}{N}\sum_{n = 0}^{N - 1} F_n e^{2\pi i k n/N}
	\end{equation}

	\pagebreak
	
	\section{Z - Transforms}
	Z-Transform usually generate a geometric series..
	
	\subsection{Revision on Geometric Series}
	
	In mathematics, a geometric series is a series with a constant ratio between successive terms. \cite{osama_2019}\\

	For example, the series
	\begin{flalign*}
		&\frac{1}{2} + \frac{1}{4} + \frac{1}{8} + \frac{1}{16} + ...
	\end{flalign*}
	is geometric, because each successive term can be obtained by multiplying the previous term by $\frac{1}{2}$\\\\
	\textbf{General formula}\\
	\begin{flalign}
		r^0 &+ r^1 + r^2 + r^3 + r^4 + ...\\\nonumber
		= 1 &+ r^1 + r^2 + r^3 + r^4 + ... = \frac{1}{1 - r}\hspace{1.5cm} ,|r| < 1
	\end{flalign}
	where $r$ is the constant ratio.\\
	
	So if $S_\infty$ is a geometric series,\\
	\begin{flalign*}
		S_\infty = \frac{First\hspace{0.2cm}Term}{1 - r}
	\end{flalign*}

	\subsection{Z-Transform General Equation}
	The unilateral Z-transform of a sequence $\left(a_k\right)_{k=0}^\infty$ is defined as
	\begin{equation}
		Z[\left(a_k\right)_{k=0}^\infty](z) = \sum_{k = 0}^{\infty} \frac{a_k}{z^k}
	\end{equation}
	or
	\begin{equation}
		f(z) = \sum_{n = 0}^{\infty} F(nT) z^{-n}
	\end{equation}
	\pagebreak
	
	\subsection{Common Z-Transforms}
	
	The following table summarize the z-transforms for some common functions
	\begin{table}[h!]
		\centering
		\begin{tabular}{|c|c|}\hline
			$a_n$ & $Z[\left(a_k\right)_{k=0}^\infty](z)$\\ \hline
			$\delta_{0n}$ & 1 \\ \hline
			$(-1)^n$ & $\frac{z}{z - 1}$\\ \hline
			$n$ & $\frac{z}{(z-1)^2}$\\ \hline
			$n^2$ & $\frac{z(z + 1)}{(z-1)^3}$\\ \hline
			$n^3$ & $\frac{z(z^2 + 4z + 1)}{(z-1)^4}$\\ \hline
			$b^n$ & $\frac{z}{z-b}$\\ \hline
			$b^n n^2$ & $\frac{bz(b + z)}{(z-b)^3}$\\ \hline
			$\cos(\alpha n)$ & $\frac{z(z - \cos(\alpha))}{z^2 - 2\hspace{0.1cm}z\cos(\alpha) + 1}$\\ \hline
			$\sin(\alpha n)$ & $\frac{z\sin(\alpha)}{z^2 - 2\hspace{0.1cm}z\cos(\alpha) + 1}$\\ \hline
		\end{tabular}
		\caption{Common Z-transform pairs}
		\label{table_fourier_trans_pairs}
	\end{table}\\\\

	\textbf{\Large Note that:} \cite{osama_2020} \\
	The discrete Fourier transform is a special case of the z-transform with
	\begin{flalign*}
		z \equiv e^{-2\pi i k/N}
	\end{flalign*}
	and a z-transform with
	\begin{flalign*}
		z \equiv e^{-2\pi i k \alpha/N}
	\end{flalign*}
	for $\alpha \neq \pm 1$ is called a fractional Fourier transform.





	\pagebreak

	\section{Exercises}
	
		Find Laplace Transform for $f(t)$:
	\begin{flalign}\nonumber
		1.\hspace{0.3cm} &f(t) = 1&
	\end{flalign}
	\begin{flalign}\nonumber
		2.\hspace{0.3cm} &f(t) = 4\cos(4t) - \sin(4t) + 2\cos(10t)&
	\end{flalign}
	\begin{flalign}\nonumber
		3.\hspace{0.3cm} &f(t) = 3\sinh(2t) + 3\sin(2t)&
	\end{flalign}
	\begin{flalign}\nonumber
		4.\hspace{0.3cm} &f(t) = e^{3t} + \cos(6t) - e^{3t}\cos(6t)&
	\end{flalign}
	\begin{flalign}\nonumber
		5.\hspace{0.3cm} &f(t) = 6e^{-5t} + e^{3t} + 5t^3 - 9&
	\end{flalign}
	\begin{flalign}\nonumber
		6.\hspace{0.3cm} &f(t) = t\cosh(3t)&
	\end{flalign}
	\begin{flalign}\nonumber
		7.\hspace{0.3cm} &f(t) = t^2\sin(2t)&
	\end{flalign}

	\pagebreak
	
		Find Inverse Laplace Transform for $F(s)$:
	\begin{flalign}\nonumber
		1.\hspace{0.3cm} &F(s) = \frac{6}{s} - \frac{1}{s-8} + \frac{4}{s-3}&
	\end{flalign}
	\begin{flalign}\nonumber
		2.\hspace{0.3cm} &F(s) = \frac{19}{s+2} - \frac{1}{3s-5} + \frac{7}{s^5}&
	\end{flalign}
	\begin{flalign}\nonumber
		3.\hspace{0.3cm} &F(s) = \frac{6s}{s^2 + 25} + \frac{3}{s^2 + 25}&
	\end{flalign}
	\begin{flalign}\nonumber
		4.\hspace{0.3cm} &F(s) = \frac{8}{3s^2 + 12} + \frac{3}{s^2 - 49}&
	\end{flalign}
	\begin{flalign}\nonumber
		5.\hspace{0.3cm} &F(s) = \frac{6s - 5}{s^2 + 7}&
	\end{flalign}
	\begin{flalign}\nonumber
		6.\hspace{0.3cm} &F(s) = \frac{1-3s}{s^2+8s+21}&
	\end{flalign}
	\begin{flalign}\nonumber
		7.\hspace{0.3cm} &F(s) = \frac{3s-2}{2s^2 - 6s - 2}&
	\end{flalign}
	\begin{flalign}\nonumber
		8.\hspace{0.3cm} &F(s) = \frac{s+7}{s^2 - 3s - 10}&
	\end{flalign}
	\begin{flalign}\nonumber
		9.\hspace{0.3cm} &F(s) = \frac{2-5s}{(s-6)(s^2+11)}&
	\end{flalign}
	\begin{flalign}\nonumber
		10.\hspace{0.3cm} &F(s) = \frac{25}{s^3(s^2+4s+5)}&
	\end{flalign}
	\begin{flalign}\nonumber
		11.\hspace{0.3cm} &F(s) = \frac{86s-78}{(s+3)(s-4)(5s-1)}&
	\end{flalign}

	\pagebreak

		Find Fourier series of $f(x)$:
	\begin{flalign}\nonumber
		1.\hspace{0.3cm} &f(x) = x\hspace{0.3cm}, \hspace{2cm} -\pi\leq x \leq \pi&
	\end{flalign}
	\begin{flalign}\nonumber
		2.\hspace{0.3cm} &f(x) =
		\left\{
		\begin{array}{ll}
			0  & \mbox{if } -\pi\leq x \leq 0\\
			\pi & \mbox{if }  \hspace{0.5cm}0 \leq x \leq \pi
		\end{array}
		\right.&
	\end{flalign}
	\begin{flalign}\nonumber
		3.\hspace{0.3cm} &f(x) =
		\left\{
		\begin{array}{ll}
			-\frac{\pi}{2}  & \mbox{if } -\pi\leq x \leq 0\\
			\frac{\pi}{2} & \mbox{if }  \hspace{0.5cm}0 \leq x \leq \pi
		\end{array}
		\right.&
	\end{flalign}
	\begin{flalign}\nonumber
		4.\hspace{0.3cm} &f(x) =
		\left\{
		\begin{array}{ll}
			0  & \mbox{if } -2\leq x \leq 0\\
			x & \mbox{if }  \hspace{0.5cm}0 \leq x \leq 2
		\end{array}
		\right.&
	\end{flalign}
	Find the Fourier cosine series
	\begin{flalign}\nonumber
		5.\hspace{0.3cm} &f(x) = x, \hspace{2cm} x\in[0,\pi]&
	\end{flalign}
	Find the Fourier sine series
	\begin{flalign}\nonumber
		6.\hspace{0.3cm} &f(x) = 1, \hspace{2cm} x\in[0,\pi]&
	\end{flalign}
	Find the Fourier sine series
	\begin{flalign}\nonumber
		7.\hspace{0.3cm} &f(x) = cos(x), \hspace{2cm} x\in[0,\pi]&
	\end{flalign}
	
	\pagebreak
	
		Find Z-Transform $(z[x_n])$ for the sequence $x_n$:
	\begin{flalign}\nonumber
		1.\hspace{0.3cm} &x_n = (\frac{1}{2})^n&
	\end{flalign}
	\begin{flalign}\nonumber
		2.\hspace{0.3cm} &x_n = n&
	\end{flalign}
	\begin{flalign}\nonumber
		3.\hspace{0.3cm} &x_n = x(nT) =
		\left\{
		\begin{array}{ll}
			1  & \mbox{if } n \geq 0 \\
			0 & \mbox{if } n < 0
		\end{array}
		\right.&
	\end{flalign}
	\begin{flalign}\nonumber
		4.\hspace{0.3cm} &x_n = e^{-ant}&
	\end{flalign}
	\begin{flalign}\nonumber
		5.\hspace{0.3cm} &x_n = x(nT) = a^n \cos\left(\frac{n\pi}{2}\right)&
	\end{flalign}
	


\pagebreak

		Prove That:
	\begin{flalign}\nonumber
		1.\hspace{0.3cm} &L[c] = \frac{c}{s}&
	\end{flalign}
	\begin{flalign}\nonumber
		2.\hspace{0.3cm} &L[e^{at}] = \frac{1}{s - a}&
	\end{flalign}
	\begin{flalign}\nonumber
		3.\hspace{0.3cm} &L[t^n] = \frac{n!}{s^{n+1}}&
	\end{flalign}
	\begin{flalign}\nonumber
		4.\hspace{0.3cm} &\int_{-\pi}^{\pi} \sin(nx)\,dx = 0&
	\end{flalign}
	\begin{flalign}\nonumber
		5.\hspace{0.3cm} &\int_{-\pi}^{\pi} \cos(nx)\,dx = 0&
	\end{flalign}
	\begin{flalign}\nonumber
		6.\hspace{0.3cm} &\int_{-\pi}^{\pi} \sin(mx)\cos(nx)\,dx = 0&
	\end{flalign}
	\begin{flalign}\nonumber
		7.\hspace{0.3cm} &\int_{-\pi}^{\pi} \cos(nx)\cos(nx)\,dx = 0&
	\end{flalign}
	\begin{flalign}\nonumber
		8.\hspace{0.3cm} &\int_{-\pi}^{\pi} \sin(mx)\sin(nx)\,dx = 0&
	\end{flalign}
	\begin{flalign}\nonumber
		9.\hspace{0.3cm} &a_0 = \frac{1}{2\pi} \int_{-\pi}^{\pi} f(x)\,dx&
	\end{flalign}
	\begin{flalign}\nonumber
		10.\hspace{0.3cm} &a_n = \frac{1}{\pi} \int_{-\pi}^{\pi} f(x)\cos(nx)\,dx = 0&
	\end{flalign}
	\begin{flalign}\nonumber
		11.\hspace{0.3cm} &b_n = \frac{1}{\pi} \int_{-\pi}^{\pi} f(x)\sin(nx)\,dx = 0&
	\end{flalign}
	\begin{flalign}\nonumber
		11.\hspace{0.3cm} &F_x[f^n(x)](k) = (2\pi ik)^n F_x[f(x)](k)&
	\end{flalign}


	\pagebreak
	\bibliography{references}
	\bibliographystyle{unsrt}
\end{document}